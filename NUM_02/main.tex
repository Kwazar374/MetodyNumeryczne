\documentclass[a4paper, 12pt]{article}
\usepackage[utf8]{inputenc}
\usepackage{amsmath}
\usepackage{amsfonts}
\usepackage{geometry}
\usepackage[T1]{fontenc}
\usepackage{hyperref}
\geometry{margin=1in}

% Tytuł i autor
\title{Raport z ćwiczenia NUM2\\Analiza uwarunkowania układów równań liniowych}
\author{Bartosz Satoła}
\date{30.10.2024}

\begin{document}

\maketitle
\tableofcontents
\newpage

\section{Cel ćwiczenia}
Celem ćwiczenia było rozwiązanie układów równań macierzowych \( A_i y = b \) dla dwóch zadanych macierzy \( A_1 \) i \( A_2 \), a także analiza wpływu zaburzenia wektora wyrazów wolnych \( b \) na stabilność rozwiązań. Dodatkowo sprawdzono, jak wprowadzenie małego zaburzenia \( \Delta b \), losowanego tak, aby jego norma była rzędu \( 10^{-6} \), wpływa na różnice w wynikach, co pozwala na ocenę uwarunkowania numerycznego macierzy.

\section{Opis ćwiczenia}
W ćwiczeniu rozwiązano układy równań liniowych:
\[
A_i y = b, \quad \text{dla } i = 1, 2
\]
gdzie wektor \( b \) jest dany jako:
\[
b = \begin{pmatrix} -2.8634904630 \\ -4.8216733374 \\ -4.2958468309 \\ -0.0877703331 \\ -2.0223464006 \end{pmatrix}.
\]
Następnie dodano do \( b \) losowe zaburzenie \( \Delta b \) o małej normie euklidesowej, a układy rozwiązano ponownie dla \( b + \Delta b \):
\[
A_i y = b + \Delta b.
\]
Wyniki dla macierzy \( A_1 \) i \( A_2 \) porównano, badając różnice między rozwiązaniami układów z pierwotnym \( b \) oraz zaburzonym \( b + \Delta b \).

\section{Wstęp teoretyczny}
\subsection{Uwarunkowanie macierzy}
Uwarunkowanie macierzy jest miarą wrażliwości rozwiązania układu równań liniowych \( Ax = b \) na małe zmiany w danych wejściowych, tj. macierzy \( A \) lub wektora \( b \). Jeżeli niewielkie zaburzenie danych prowadzi do dużych zmian w rozwiązaniu \( x \), układ nazywamy \textit{źle uwarunkowanym}. 

Z kolei dobrze uwarunkowane układy charakteryzują się stabilnością numeryczną, co oznacza, że małe błędy w danych (np. wynikające z błędów zaokrągleń lub zakłóceń w \( b \)) wywołują proporcjonalnie małe zmiany w wynikach.

\subsection{Współczynnik uwarunkowania}
Współczynnik uwarunkowania macierzy, oznaczany przez \( \kappa(A) \), jest wyrażony jako:
\[
\kappa(A) = ||A|| \cdot ||A^{-1}||
\]
gdzie \( || \cdot || \) oznacza normę macierzy. Wartość \( \kappa(A) \) mówi o stabilności układu:
\begin{itemize}
    \item Jeśli \( \kappa(A) \approx 1 \), układ jest dobrze uwarunkowany.
    \item Im wyższe \( \kappa(A) \), tym bardziej układ jest podatny na błędy, a zatem gorzej uwarunkowany.
\end{itemize}
W praktyce, dobrze uwarunkowane układy zapewniają stabilność obliczeń i dokładność wyników, co jest szczególnie ważne w zastosowaniach, gdzie precyzja jest kluczowa.

\section{Omówienie wyników}
Po rozwiązaniu układów dla obu macierzy \( A_1 \) i \( A_2 \) oraz dla wektora \( b \) z perturbacją \( \Delta b \), uzyskano następujące obserwacje:
\begin{itemize}
    \item Dla macierzy \( A_1 \) rozwiązanie układu wykazało mniejsze odchylenie po wprowadzeniu perturbacji, co świadczy o jej lepszym uwarunkowaniu.
    \item Macierz \( A_2 \), charakteryzująca się wyższym współczynnikiem uwarunkowania, wykazała większą wrażliwość na zaburzenie \( \Delta b \), co potwierdza, że układ jest gorzej uwarunkowany.
    \item Wyniki te są zgodne z oczekiwaniami teoretycznymi, które wskazują, że macierze o wyższym współczynniku uwarunkowania są bardziej podatne na błędy związane z małymi perturbacjami danych.
\end{itemize}

\section{Wnioski}
Z przeprowadzonego ćwiczenia można wyciągnąć następujące wnioski:
\begin{itemize}
    \item Układy równań liniowych dla dobrze uwarunkowanych macierzy są stabilniejsze i mniej podatne na błędy, nawet w obecności zaburzeń.
    \item Współczynnik uwarunkowania macierzy jest kluczowym wskaźnikiem stabilności układu równań. Wyższa wartość \( \kappa(A) \) oznacza większą podatność na błędy, co może prowadzić do znaczących zmian w rozwiązaniach.
    \item W przypadku zastosowań wymagających wysokiej dokładności, zalecane jest stosowanie macierzy dobrze uwarunkowanych lub technik, które poprawiają stabilność obliczeń.
\end{itemize}

\end{document}
